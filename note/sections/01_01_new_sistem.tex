\subsection{Необходимость разработки сервиса}

Громоздкость отечественного бумажного делопроизводства давно уже стала притчей во языцех. Российский бюрократизм взращивался в течение десятков, а то и сотен лет, и путь документа через всю вертикаль власти от создания до его согласования и принятия в работу до сих пор занимает не один день. Этот бюрократизм нашел свое продолжение и в частном бизнесе. На «этажах» иерархии документы нередко теряются, срываются сроки их исполнения, а в результате – резко снижается эффективность управления, уходят клиенты. Для того чтобы исправить такое положение, и были разработаны системы электронного документооборота (СЭД).

Не каждый бизнесмен внедряет систему электронного документооборота. Поэтому важной функцией СЭД является возможность интеграции бумажных документов, полученных от различных структур. Вид документов может быть разнообразный, например:
\begin{itemize}
  \item информационно-справочные документы;
  \item организационно-распорядительные документы;
  \item приказ;
  \item акт;
  \item протокол;
  \item письмо;
  \item справка;
  \item объяснительная записка;
  \item докладная записка.
\end{itemize}

