\subsection{Общие требования}


\begin{itemize}
  \item[1] Назначение разработки.

  \hspace*{2.5em}Итогом дипломного проектирования должен быть проект, позволяющий пользователям добавлять документы в СЭД путем их сканирования.Продукт должен быть интересен большому и среднему бизнесу, который хочет сократить время на регистрацию бумажных документов в СЭД.

  \item[2] Входные данные.

	\hspace*{2.5em}Входные данные получаемые от пользователя:
  \begin{itemize}
  	\item информация о файлах, добавляемых в результате сканирования (название, тип, режим добавления);
  	\item пользователь должен выбрать доступный сканер из списка;
  	\item настройки для сканирования (разрешение сканирования, размер страниц и т.д).
  \end{itemize}

  \hspace*{2.5em}Входные данные получаемые через customProtocol:
  \begin{itemize}
  	\item информация о пользователе; 
  	\item информация о регистрационной карточке;
   	\item информация требуемая для прикрепления документа к регистрационной карточке (идентификатор карточки, идентификатор пользователя);
  	\item некоторые технические значения, получаемые через customProtocol (расположение WCF сервиса, идентификатор билета для доступа к сервису и т.д).
  \end{itemize}

  \hspace*{2.5em}Информация о пользователе и регистрациооной карточке должны указывать на существующие, соответствующие записи в системе. Технические значения должны быть валидны.
 
  \item[4] Требования к временным характеристикам.

  \hspace*{2.5em}Как правило, осуществляется сканирование большого числа документов, это может занять некоторое время. Для повышения эффективности, обработка изоображений полученных от сканера, а так же их последующее совмещение в единый файл должно занимать как можно меньше времени. 

  \hspace*{2.5em}Получения изоображений от сканера должно занимать менее 0.4 секунды за одну страницу. Под получением подразумевается отображение странички на предпросмотр и способность приступить к редактированию. За эталонное расширение считать 300dpi.

  \hspace*{2.5em}При формирование итогового файла, время, пройденное с момента нажатия кнопки «Прикрепить» до начала загрузки файла должно удовлетворять следующим условиям:
	\begin{itemize}
		\item для формата pdf – менее 0.6 секунд за страницу;
		\item для формата djvu – менее 0.9 секунд за страницу;
		\item для формата TIFF – менее 0.4 секунд за страницу.
	\end{itemize}
  \item[5] Требования к надежности.

  \hspace*{2.5em}Система должна работать без сбоев, так как сбой может стать причиной для повторного сканирования большого числа документов.

  \item[6] Требования к безопасности.

  \hspace*{2.5em}Программа не должна оставлять никаких следов на компьютере после сканирования. Хранение временных данных (отсканированных документов, информации о пользователе, регистрационной карточке и т.д) в любом виде, после завершения операции прикрепления, недопустимо.

  \hspace*{2.5em}Прикрепление файлов должно осуществляться точно к запрошенным документам, иначе отсканированные документы  могут попасть к пользователям неуполномоченным на просмотр данного документа.
  
  \item[7] Требования к операциооной системе.

  \hspace*{2.5em}Для поддержки наибольшего количества пользователей была выбрана платформа Windows. Приложение совместимо со следующими операционными системами: Windows 7, Windows 8, Windows 8.1, Windows 10.

\end{itemize}