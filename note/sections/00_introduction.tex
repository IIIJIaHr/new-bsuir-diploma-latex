%Дописать введение про сервера
\sectioncentered*{ВВЕДЕНИЕ}
\addcontentsline{toc}{section}{ВВЕДЕНИЕ}

В наши дни большинство документов изначально создаются в электронном виде. Те же, что попадают в организацию на бумаге, часто подвергаются оцифровке. Поэтому, когда мы говорим об управлении документооборотом, то должны иметь в виду не только бумажные документы, но и формализацию движения электронных версий, то есть электронный документооборот. 

Виды и способы организации электронного документооборота могут быть разными – создание общего файлового хранилища на сервере, использование внутренней почты или иных коммуникационных систем. Если идти дальше, нужно внедрять схему, которая позволит упорядочить работу и с бумажными документами, и с электронными. Это позволяют сделать системы электронного документооборота различных видов.

Времена, когда внедрение СЭД могло быть данью моде, ушли в прошлое. Сейчас организации, переходящие на электронный документооборот, в первую очередь думают об эффективности. Повышение эффективности возможно двумя способами – через увеличение результата и уменьшение затрат. Современные СЭД используют оба эти способа.

Таким образом, поскольку истинной целью системы электронного документооборота является повышение эффективности документооборота в организации, следует бороться за каждую рутинную операцию. Для дальнейшей автоматизации данных действий.

В веб версии СЭД. Для добавления бумажных документов в систему СЭД обычно регистратору требуется использовать стороннюю программу для сканирования, сохранить результат сканирования во временную директорию, преобразовать документы в нужный формат, в требуемой последовательности. Затем средствами СЭД добавить результирующие, оцифрованные документы к регистрационной карточке документа.

Автоматизацией данного процесса может послужить десктопное приложение, связанное с системой электронного документооборота. Приложение будет устанавливаться на компьютер. Такое решение позволит по нажатию кнопки в СЭД вызвать ПО, установленное на компьютере, в настройках которого будет задаваться формат результирующего документа. И в дальнейшем взаимодействие со сканером осуществляется через данное приложение. Так же будет возможность выбора определенных листов документа.

С учетом вышеизложенного для добавления бумажных документов в СЭД потребуется лишь несколько кликов.