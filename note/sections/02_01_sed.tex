\subsection{Системы электронного документооборота}
\label{sec:analysis:serv_arch}

Официального определения системы электронного документооборота (СЭД), утвержденного стандартом, не существует. По сути, СЭД — это система, позволяющая автоматизировать основные процедуры делопроизводства компании. Она охватывает процессы создания, обработки, тиражирования, передачи, хранения документов, контроля над их исполнением и предназначена для эффективного управления предприятием.


\begin{figure}[h!]
	\centering
	\includegraphics[scale=0.78]{sed_in_russia_today.png}
	\caption{Схема использования СЭД}
\end{figure}

При внедрении СЭД на предприятии обычно ставят следующие цели:
\begin{itemize}
	\item сокращение или полный отказ от бумажного документооборота; 
	\item создание единой информационной базы компании; 
	\item снижение риска утери документа; 
	\item структурирование всей документации по утвержденной номенклатуре;
	\item повышение дисциплины среди сотрудников благодаря возможности отслеживания деятельности исполнителя конкретного документа;
	\item контроль над исполнением документов в соответствии с резолюциями руководителя;
	\item повышение эффективности работы компании.
\end{itemize}

СЭД — один из главных информационных ресурсов компании, который используется для работы с самыми разными видами и типами документов, интегрируется с другими деловыми системами. Традиционный для СЭД функционал, по мнению экспертов компании «1С», расширяется в направлении автоматизации совместной работы сотрудников.

Как и любой инструмент, СЭД нужно применять по назначению и тогда, когда это действительно необходимо компании. Только в этом случае внедрение системы будет эффективно и принесет реальную пользу.

Определить эффективность внедрения СЭД в цифровом выражении достаточно сложно. Компанией DIRECTUM на основе собственной методики были выделены ключевые эффекты от внедрения СЭД на предприятии \cite{directum}. При этом величина эффекта зависит от того, является ли компания крупным предприятием с несколькими филиалами или небольшим предприятием, расположенном в одном офисе. Потенциальные выгоды следующие:

\begin{itemize}
  \item[1] \textbf{Снижение материальных затрат:}
	
	\begin{itemize}
	  \item небольшое предприятие — на 5\%
	  \item крупное предприятие с несколькими филиалами — на 20\%
	\end{itemize}

  \item[2] \textbf{Экономия на базовых процессах} — исходящие и входящие документы, организационно-распорядительный документооборот, контроль исполнения поручений:
	
	\begin{itemize}
	  \item небольшая компания — от 8 до 15\%;
	  \item крупное предприятие — до 50\%.
	\end{itemize}
	
  \item[3] \textbf{Экономия на конкретных операциях}, не привязанных к процессам, — поиск документов, обеспечение доступа к ним и т.д. — от 3 до 24\% в зависимости от стиля работы с документами и от организации системы их хранения. Если создать общедоступное хранилище электронных документов с четко прописанными регламентами работы и хранения, правами доступа к ним, то эффект для компании будет максимальным.

  \item[4] \textbf{Снижение рисков.} Этот эффект касается стратегических показателей и меньше всего поддается формальному расчету. В некоторых случаях СЭД позволяет снизить риски просрочки согласования и заключения договоров до 60\%.

\end{itemize}

По оценкам экспертов, внедрение СЭД на предприятии окупается в среднем за срок от 3 месяцев до 3 лет \cite{ecm_journal}.
