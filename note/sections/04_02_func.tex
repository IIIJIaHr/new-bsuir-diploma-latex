\subsection{Функциональные требования}
\label{sub:requirements:func_rec}

Функциональные требования объясняют, что должно быть сделано. Они идентифицируют задачи или действия, которые должны быть выполнены. Функциональные требования определяют действия, которые система должна быть способной выполнить.

К разрабатываемому приложению выдвигаются следующие функциональные требования.
% \begin{enumerate}

\subsubsection{Главное окно приложения }
\label{sub:requirements:func_rec:main_window}

Окно должно содержать следующие элементы:
\begin{itemize}
	\item Область главного предпросмотра. Располагается по центру. Здесь отображается выделенная страница;
	\item Область со всеми отсканированными страницами в миниатюре. Располагается слева;
	\item Кнопки управления страницей(повороты на 90, 180, 270 градусов). Располагаются под окном предпросмотра;
	\item Над главной областью предпросмотра должны отображаться: имя пользователя, осуществляющего сканирование и номер регистрационной карточки к которой будет прикреплён документ;
	\item Кнопку «Отвязать сессию»;
	\item Кнопку «Выделить все»;
	\item Кнопку «Удалить»;
	\item Кнопку «Сканировать»;
	\item Кнопку «Настройки»;
	\item Кнопку «Отправить»;

\end{itemize}

Область главного предпросмотра: 
\begin{enumerate}
	\item В области отображается заставка в случаях: отсканированных страниц нет, либо не выделена ни одна страница;
  	\item При выделении страницы, она должна отобразиться на предпросмотр;
 	\item При наведении курсора на изображение, должно быть доступно маштабирование, осуществляемое колесиком мышки;
\end{enumerate}

Перечень страниц: 
\begin{enumerate}
	\item Итоговый документ должен собираться из страниц, выделенных в перечне, в соответствующем порядке; 
	\item Должен быть доступен функционал изменения очередности страниц в перечне(перетаскивание зажатием левой кнопки мыши); 
	\item Если определенная страница или несколько страниц были удалены, они должны отсутствовать в перечне;
\end{enumerate}

Требования к кнопкам:
\begin{enumerate}
	\item Кнопка «Отправить» осуществляет соединение выделенных страниц, и отправляет цельным документом итоговый файл в СЭД.
	\item Кнопка «Удалить» удаляет страницы из итогового документа, перечня страниц и предпросмотра, если страница находилась на предпросмотре;
	\item Кнопка «Выделить все» выделяет все имеющиеся страницы в перечне;
	\item Кнопка «Сканировать» открывает интерфейс драйвера сканера и начинает сканирование;
	\item По нажатию на кнопку «Отвязать сессию» имя пользователя и номер карточки должны очищаться. Кнопка должна быть в виде иконки;
	\item Кнопка «Настройки» открывает меню настроек;
	\item Кнопки поворота должны осуществлять поворот выделенной страницы на соответствующий градус;
	\item Кнопка «Поворот на 180 градусов» располагается между кнопками «Поворот на 90 градусов» и «Поворот на 270 градусов»;
	\item Если текущая сессия привязана к какой-либо карточке, кнопка отвязки сессии должна быть неактивна;
	\item Кнопка «Отвязать сессию» должна быть неактивна если пользователь не привязан;
	\item Кнопка «Удалить» должна быть неактивна если выделенные страницы отсутствуют, либо перечень страниц пуст. Кнопка должна быть в виде иконки;
	\item Иконка кнопки «Поворот на 90 градусов» должна выглядеть в виде стрелки, повернутой влево;
	\item Иконка кнопки «Поворот на 270 градусов» должна выглядеть в виде стрелки, повернутой вправо;
	\item Иконка кнопки «Поворот на 180 градусов» должна выглядеть в виде двух стрелок, указывающих на концы друг друга;
\end{enumerate}


\subsubsection{Окно настроек}
\label{sub:requirements:func_rec:settings_window}


Окно настроек разграничивается в виде вкладок. Должны быть доступны следующие вкладки:
\begin{enumerate}
	\item Вкладка «Сканирование»;
	\item Вкладка «Общие»;
	\item Вкладка «Выгрузка»;
	\item Вкладка «Дополнительные»;
\end{enumerate}

На вкладке «Сканирование» должна осуществляться настройка следующего функционала:
\begin{enumerate}
	\item Список подключенных к компьютеру сканеров. Должна быть возможность выбора сканера из данного списка. Выбранный сканер будет осуществлять сканирование документов;
	\item Возможность выбора типа подключения сканера. TWAIN или WIA;
	\item Возможность включить либо выключить запуск стандартного окна настройки драйвера сканера. Будет запускаться после нажатия кнопки «Сканировать»;
	\item Возможность включения, выключения функции «Автоматический поворот»;
	\item Возможность включения, выключения функции «Автоматическое распознование границ страницы»;
	\item Настройка способа сканирования: сканировать все страницы, сканировать определенное число страниц, одностороннее сканирование, двустороннее сканирование. Для выбора количества страниц, пользователю должна быть предоставлена возможность ввести число с клавиатуры;
	\item Настройка размера страницы. Выбирается из списка. В списке отображаются поддерживаемые выбранным сканером значения;
	\item Установка ориентации: альбомная или портретная;
	\item Установка разрешения сканирования. Для установки разрешения должна быть возможность изменить единицы измерения, которые выбираются из выпадающего списка. Отдельно устанавливается разрешение по-вертикали и по-горизонтали. Доступные значения предлагаются в выпадающем списке. Список формируется на основе возможностей выбранного сканера;
	\item Установка цвета сканирования: черно-белая или цветная. Если печать черно-белая, должна быть возможность выбора глубины цвета в битах. Глубина цвета выбирается из выпадающего списка;
\end{enumerate} 

Поскольку большинство значений по настройке сканера не известны до момента выбора сканера, возможность изменения таких настроек следует исключить. При открытии вкладки «Сканирование», список доступных сканеров должен быть доступен всегда. Если сканер не выбран, остальные настройки должны быть неактивны.

Если у сканера отсутствует возможность настройки какого-либо поведения при сканировании, эта настройка должна быть неактивна.

На вкладке «Общие» должен отображаться функционал настройки работы приложения, не связанный с сканированием:
\begin{enumerate}
	\item Показывать ли иконку приложения;
	\item Включить либо выключить автозапуск после включения компьютера;
\end{enumerate}

На вкладке «Выгрузка» должна быть настройка конфигурации прикрепляемых документов:
\begin{enumerate}
	\item Текстовое поля для ввода префикса;
	\item Выпадающий список с возможными типами документа;
	\item Возможность отключения дополнительного окна для ввода имени файла;
\end{enumerate}

\subsubsection{Требования к поведению системы}
\label{sub:requirements:func_rec:requirements_system}

Система должна прикреплять отсканированные документы двумя способами:
\begin{itemize}
	\item Одностраничный;
	\item Мультистраничный;
\end{itemize}

Требования к одностраничному методу:
\begin{enumerate}
	\item Каждая страница добавляется как отдельный документ.
	\item Имя документа определяется исходя из префикса в настройках и уникального идентификатора.
\end{enumerate}

Требования к мультистраничному методу:
\begin{enumerate}
	\item Все выделенные страницы добавляются в соответствующем порядке в один документ.
	\item После нажатия кнопки «Отправить» должно открываться окно, в котором пользователь введет имя документа. Под этим именем документ будет добавлен в СЭД.
	\item Если в настройках пользователь не отметил галку «Спрашивать имя», окно не должно отображаться. В таком случае имя файла устанавливается исходя из префикса в настройках.
\end{enumerate}

У пользователя должна быть возможность выбрать тип результирующего документа. Поддерживаемые типы:
\begin{itemize}
	\item JPG;
	\item PNG;
	\item PDF;
	\item TIFF;
	\item DJVU;
	\item Мультистраничный TIFF;
	\item Мультистраничный PDF;
\end{itemize}

Действия после того как документ был отправлен в СЭД:
\begin{enumerate}
	\item Страницы, вошедшие в итоговый документы должны удалиться из перечня страниц; 
	\item Окно предпросмотра должно обновиться с соответствующей страницей; 
	\item Кнопка «Отправить» должна стать неактивной;
	\item Текущая сессия должна быть закрыта, пользователь и номер документа очищаются.
\end{enumerate}

% \end{enumerate}