\subsection{JavaScript}

JavaScript изначально создавался для того, чтобы добавить интерактивность на веб-сайт (например: игры, отклик при нажатии кнопок или при вводе данных в формы, динамические стили, анимация). Программы на этом языке называются скриптами. В браузере они подключаются напрямую к HTML и выполняется сразу после загрузки страницы.

Основные архитектурные черты языка: динамическая типизация, слабая типизация, автоматическое управление памятью, прототипное программирование, функции как объекты первого класса.

JavaScript может выполняться не только в браузере, а где угодно, нужна лишь специальная программа – интерпретатор. Процесс выполнения скрипта называют «интерпретацией».

Во все основные браузеры встроен интерпретатор JavaScript, именно поэтому они могут выполнять скрипты на странице. Но, разумеется, JavaScript можно использовать не только в браузере. Это полноценный язык, программы на котором можно запускать и на сервере, и даже в стиральной машинке, если в ней установлен соответствующий интерпретатор.